\documentclass{beamer}

\usepackage[utf8]{inputenc}
\usepackage{skak}
\usepackage{listings}
\usepackage{xcolor}

\usepackage[
backend=biber,
style=alphabetic,
citestyle=authortitle-icomp
]{biblatex}

\addbibresource{presentation.bib}

\usetheme{metropolis}

\title{Chesskell: Modelling a Two-Player Game at the Type-Level}
\author{Toby Bailey}
\institute{Department of Computer Science}
\date{\today}

\definecolor{background}{rgb}{0.92, 0.92, 0.92}
\definecolor{comments}{rgb}{0.0, 0.64, 0.0}
\definecolor{keywords}{rgb}{0.0, 0.0, 0.64}
\definecolor{identifiers}{rgb}{0.63, 0.81, 0.94}
\definecolor{strings}{rgb}{1.0, 0.3, 0.0}

\lstset{
    language=haskell,
    basicstyle=\footnotesize\ttfamily,
    backgroundcolor=\color{background},
    keywordstyle=\color{keywords}\bfseries,
    commentstyle=\color{comments}\textit,
    stringstyle=\color{strings},
    % identifierstyle=\color{identifiers},
    breakatwhitespace=true,
    breaklines=true,
    keepspaces=true,
    captionpos=b,
    frame=tlbr,    % Margin at all 4 sides
    framesep=4pt,  % Margin size
    framerule=0pt,
    morekeywords={Eval, Exp, family, instance},
    deletekeywords={map, and, error, take}
}

% Custom command for all inline code styling
\newcommand{\inline}[1]{\lstinline[basicstyle=\ttfamily]{#1}}

\begin{document}

\frame{\titlepage}

\begin{frame}{Why do type systems exist?}
Type systems exist because: \pause we want to avoid errors (\cite{cardellitypes}).

\pause But not domain-specific logical errors?

\pause Solution: model your domain in the types!
\end{frame}

\begin{frame}{Why Chess?}

\begin{itemize}
    \item<1-3> It's popular and internationally known;
    \item<2-3> It's been widely studied in the field of Computer Science (\cite{chesseducation}, \cite{chessml});
    \item<3-4> It has a \emph{well-defined ruleset}.
\end{itemize}
    
\end{frame}

\begin{frame}{A note on Chess}

There are two \emph{Teams}; Black and White.

\pause

Each Team has 16 \emph{Pieces}; 8 Pawns, 2 Rooks, 2 Bishops, 2 Knights, a Queen, and a King.

\pause

Each piece has different movement rules, allowing them to move around the 8x8 board.

\pause

Pieces can remove other pieces from the board via \emph{capture}; which almost always involves moving to the other piece's square.

\end{frame}

\begin{frame}[fragile]{A Short Example}

Below is a valid move by a White Pawn:

\begin{figure}[h]
    \centering
    \newgame
    \scalebox{0.55}{\showboard}
    \quad
    \hidemoves{1. e4}
    \scalebox{0.55}{\showboard}
    \label{validpawnmove}
\end{figure}

\pause

\begin{lstlisting}
chess
    pawn e2 to e4
end
\end{lstlisting}

\end{frame}

\begin{frame}[fragile]{A Short Example cont.}

Below is an \emph{invalid} move by a White Pawn:

\begin{figure}[h]
    \centering
    \newgame
    \scalebox{0.55}{\showboard}
    \quad
    \hidemoves{1. e5}
    \scalebox{0.55}{\showboard}
    \label{badpawnmove}
\end{figure}

\pause

\begin{overprint}

\onslide<2>\begin{lstlisting}
chess
    pawn e2 to e5
end
\end{lstlisting}

\onslide<3>\begin{lstlisting}
-- Fails to compile with type error:
--     * There is no valid move from E2 to E5.
--     The Pawn at E2 can move to: E3, E4
chess
    pawn e2 to e5
end
\end{lstlisting}

\end{overprint}

\end{frame}

\begin{frame}[fragile]{A Little Terminology}

In Haskell, values have \emph{types}, and types have \emph{kinds}.

\pause

Luckily, we can \emph{promote} types to kinds with the \inline{-XDataKinds} extension (\cite{givingpromotion}).

\pause

\begin{lstlisting}
data Book = Fiction | NonFiction
\end{lstlisting}

With the extension, this creates the values \inline{Fiction} and \inline{NonFiction} of type \inline{Book}, and also the \emph{types} \inline{'Fiction} and \inline{'NonFiction} of kind \inline{Book}.

\end{frame}

\begin{frame}[fragile]{A Little Terminology cont.}

In Haskell, you compute on values with \emph{functions}.

\begin{lstlisting}
factorial :: Int -> Int
factorial 0 = 1
factorial x = x * factorial (x - 1)
\end{lstlisting}

\pause

But you have to use \emph{type families} to compute on types (\cite{opentfs}, \cite{closedtfs}).

\begin{lstlisting}
type family Factorial (x :: Nat) :: Nat where
    Factorial 0 = 1
    Factorial x = Mult x (Factorial (x - 1))
    
type family Mult (x :: Nat) (y :: Nat) :: Nat where
    Mult 0 y = 0
    Mult 1 y = y
    Mult x y = y + (Mult (x - 1) y)
\end{lstlisting}

\end{frame}

\begin{frame}[fragile]{Problems with Type Families?}

Lots of idiomatic Haskell code relies on functions being \emph{first-class}; partial application, mapping, etc.

\begin{lstlisting}
x = map (+ 2) [1,2,3]
\end{lstlisting}

\pause

But type families can't be partially applied!

\begin{lstlisting}
-- Type error: type family (+) was expecting 2 arguments, got 1
type X = Map (+ 2) '[1,2,3]
\end{lstlisting}
    
\end{frame}

\begin{frame}[fragile]{Introducing First Class Families}

Thanks to Li-yao Xia, we have First Class Families!

It relies on a data type \inline{Exp}, and a type family \inline{Eval}, to create a type-level interpreter:

\begin{lstlisting}
type Exp a = a -> *
type family Eval (e :: Exp a) :: a
\end{lstlisting}

% You define a new \emph{data type} to hold the arguments, where the return types are wrapped in \inline{Exp}, and an \inline{Eval} instance to define the behaviour of the function.

\end{frame}

\begin{frame}[fragile]{Making a First Class Family}

\begin{lstlisting}
type family And (x :: Bool) (y :: Bool) :: Bool where
    And True  True  = True
    And True  False = False
    And False True  = False
    And False False = False
\end{lstlisting}

\pause

becomes:

\begin{lstlisting}
data And :: Bool -> Bool -> Exp Bool
type instance Eval (And True  True)  = True
type instance Eval (And True  False) = False
type instance Eval (And False True)  = False
type instance Eval (And False False) = False
\end{lstlisting}

\end{frame}

\begin{frame}[fragile]{Creating the Types for Chess}

Using promotion (as we explain earlier), we define appropriate types for use with type families in Chess. We give some examples below, in both regular and GADT (\cite{gadts}) syntax:

\begin{lstlisting}
data Team = Black | White

data PieceInfo where
    Info :: Nat -> Position -> PieceInfo

data Piece where
    MkPiece :: Team -> PieceName -> PieceInfo -> Piece
\end{lstlisting}

\end{frame}

\begin{frame}[fragile]{The Board type}

To avoid repeated length checks, we use \emph{length-indexed vectors} with a type-level implementation of Peano natural numbers:

\begin{lstlisting}
data Vec (n :: Nat) (a :: Type) where
    VEnd   :: Vec Z a
    (:->)  :: a -> Vec n a -> Vec (S n) a
\end{lstlisting}

\pause

Since a Chess board is always an 8x8 grid, we use vectors of vectors:

\begin{lstlisting}
type Eight = (S (S (S (S (S (S (S (S Z))))))))
type Row   = Vec Eight (Maybe Piece)
type Board = Vec Eight Row
\end{lstlisting}

\end{frame}

\begin{frame}[fragile]{The BoardDecorator type}

In the codebase, we use a wrapper data structure to hold helpful information along with the \inline{Board} for rule checking:

\begin{lstlisting}
data BoardDecorator where
    Dec :: Board
        -> Team
        -> Position
        -> (Position, Position)
        -> Nat
        -> BoardDecorator
\end{lstlisting}

\end{frame}

\begin{frame}[fragile]{Representing Movement}

% In Chesskell, we represent movement with a single First Class Family that performs the given movement on a \inline{BoardDecorator}:

Movement is expressed as a single First Class Family:

\begin{lstlisting}
data Move :: Position -> Position -> BoardDecorator -> Exp BoardDecorator
\end{lstlisting}

\pause

Thanks to First Class Families, we can extend this with rule-checking naturally; using a type-level version of the function composition operator, \inline{(.)}:

\begin{lstlisting}
PostMoveCheck2 . PostMoveCheck1 . Move fromPos toPos . PreMoveCheck2 . PreMoveCheck1
\end{lstlisting}

\end{frame}

\begin{frame}[fragile]{Interacting with Type-Level model at the value level}

The core idea is wrapping the \inline{BoardDecorator} type in a \inline{Proxy}, so that it can be passed around within a value by functions:

\begin{lstlisting}
data Proxy a = Proxy

edslMove :: SPosition from
         -> SPosition to
         -> Proxy (b :: BoardDecorator)
         -> Proxy (Eval (Move from to b))
edslMove (x :: SPosition from) (y :: SPosition to) (z :: Proxy (b :: BoardDecorator))
    = Proxy @(Eval (Move from to b))
\end{lstlisting}

\pause

But this would still look similar to Haskell syntax; we need a new approach.

\end{frame}

\begin{frame}[fragile]{Creating the EDSL}

Ideally, the EDSL should look like existing chess notation:

\begin{verbatim}
1. e4 e5 2. Nf3 Nc6 3. Bb5 a6
\end{verbatim}

\pause

Can achieve using Continuation Passing Style, inspired by Dima Szamozvancev's Flat Builders work (\cite{mezzo}).

\end{frame}

\begin{frame}[fragile]{Making continuations for Chess}

The core continuations are \inline{chess}, the piece continuations, and \inline{to}:

\pause

\begin{overprint}

\onslide<2>\begin{lstlisting}
type Spec t = forall m. (t -> m) -> m

chess :: Spec (Proxy StartDec)
chess cont = cont (Proxy @StartDec)
\end{lstlisting}

\onslide<3>\begin{lstlisting}
data MoveArgs where
    MA :: BoardDecorator
       -> Position
       -> PieceName
       -> Position
       -> MoveArgs

pawn :: Proxy (b :: BoardDecorator)
     -> SPosition fromPos
     -> Spec (Proxy (MA b fromPos 'Pawn))
pawn (dec :: Proxy b) (from :: SPosition fromPos) cont
    = cont (Proxy @(MA b fromPos Pawn))
\end{lstlisting}

\onslide<4>\begin{lstlisting}
to :: Proxy (MA (b :: BoardDecorator) (fromPos :: Position) (n :: PieceName))
   -> SPosition toPos
   -> Spec (Proxy (Eval (MoveWithStateCheck n fromPos toPos b)))
to (args :: Proxy (MA (b :: BoardDecorator) (fromPos :: Position) (n :: PieceName))) (to' :: SPosition toPos) cont
    = cont (Proxy @(Eval (MoveWithStateCheck n fromPos toPos b)))
\end{lstlisting}

\end{overprint}
    
\end{frame}

\begin{frame}[fragile]{Using the Chess continuations}

All of the above continuations can be chained together (along with \inline{end} which ends the continuation stream) like so:

\begin{lstlisting}
game = chess pawn a1 to a2 end
\end{lstlisting}

\end{frame}

\begin{frame}[fragile]{A Longer Example}

Below is a short game, ending in checkmate by White:

\begin{figure}[h]
    \centering
    \newgame
    \scalebox{0.55}{\showboard}
    \quad
    \hidemoves{1. e4 f5 2. Qf3 g5 3. Qh5}
    \scalebox{0.55}{\showboard}
    \label{threemovecheckmate}
\end{figure}

\pause

\begin{lstlisting}
game = chess
    pawn e2 to e4
    pawn f7 to f5
    queen d1 to f3
    pawn g7 to g5
    queen f3 to h5
end
\end{lstlisting}

\end{frame}

\begin{frame}[fragile]{A Longer Example}

What about a piece trying to move after Checkmate, when the game ends?

\begin{overprint}

\onslide<1>\begin{lstlisting}
game = chess
    pawn e2 to e4
    pawn f7 to f5
    queen d1 to f3
    pawn g7 to g5
    queen f3 to h5
    pawn g5 to g4
end
\end{lstlisting}

\onslide<2>\begin{lstlisting}
-- Below results in the following type error:
    -- * The Black King is in check after a Black move. This is not allowed.
    -- * When checking the inferred type
    --     game :: Data.Proxy.Proxy (TypeError ...)
game = chess
    pawn e2 to e4
    pawn f7 to f5
    queen d1 to f3
    pawn g7 to g5
    queen f3 to h5
    pawn g5 to g4
end
\end{lstlisting}

\end{overprint}

\end{frame}

\begin{frame}[fragile]{A Longer Example cont.}

What about if the White Queen tries to move through another piece, mid-game?

\begin{overprint}

\onslide<1>\begin{lstlisting}
game = chess
    pawn e2 to e4
    pawn f7 to f5
    queen d1 to d3  -- Invalid move
    pawn g7 to g5
    queen f3 to h5
end
\end{lstlisting}

\onslide<2>\begin{lstlisting}
-- Below results in the following type error:
    -- * There is no valid move from D1 to D3.
    -- The Queen at D1 can move to: E2, F3, G4, H5, ...
    -- * When checking the inferred type
    -- game :: Data.Proxy.Proxy (...)
game = chess
    pawn e2 to e4
    pawn f7 to f5
    queen d1 to d3  -- Invalid move
    pawn g7 to g5
    queen f3 to h5
end
\end{lstlisting}

\end{overprint}

\end{frame}

\begin{frame}[fragile]{A Longer Short Example}

We also developed a shorthand syntax! 

\pause

The below game:

\begin{lstlisting}
game = chess
    pawn e2 to e4
    pawn f7 to f5
    queen d1 to f3
    pawn g7 to g5
    queen f3 to h5
end
\end{lstlisting}

\pause

becomes:

\begin{lstlisting}
game = chess
    p e4 p f5
    q f3 p g5
    q h5
end
\end{lstlisting}

\end{frame}

\begin{frame}{Testing}

Combination of:

\pause

\begin{itemize}
    \item<2-> Unit testing with assertions, based on whether a code snippet compiles or fails to compile;
    \item<3-> EDSL testing with famous Chess games, written out in Chesskell notation.
\end{itemize}
    
\end{frame}

\begin{frame}{Compile-time and memory issues}

Compile-time and memory issues came up time and again throughout development; putting a hard limit on the length of Chesskell games.

With some games, GHC will run out of memory (\textgreater 25GB) and crash.

Through testing, it seems the upper limit is \textbf{12 moves maximum}; some 12-move games tested compile, most 10-move games compile, and all 8-move games compile.
    
\end{frame}

\begin{frame}{Further Work}

\pause

\begin{itemize}
    \item<2-> More domains could be modelled at the type level, not just Chess;
    \item<3-> A session-typed version of Chesskell;
    \item<4-> Further optimisations to try and increase the move limit;
    \item<5-> An automated tool to transform from Algebraic Notation into Chesskell notation.
\end{itemize}
    
\end{frame}

\begin{frame}{Conclusions}

We have created:

\pause

\begin{itemize}
    \item<2-> A full type-level model of Chess, which enforces all rules in the FIDE 2018 Laws of Chess;
    \item<3-> An EDSL for describing Chess games and creating custom chess boards, which uses the type-level model for rule-checking;
    \item<4-> Interesting findings on compile-time and memory usage issues.
\end{itemize}

\begin{overprint}
\onslide<5>Furthermore, Chesskell is unique and has never been done before. Though there is room for further work and improvement, Chesskell is a success!
\end{overprint}

    
\end{frame}

\begin{frame}[allowframebreaks]{References}

\printbibliography
    
\end{frame}

\begin{frame}[standout]

Demo
    
\end{frame}

\begin{frame}[standout]

Q\&A
    
\end{frame}

\end{document}
