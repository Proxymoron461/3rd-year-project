\chapter{Conclusions}

The project is a success. We have presented Chesskell, a Haskell EDSL for describing Chess games where a full encoding of the FIDE 2018 Laws of Chess in the types rules out illegal moves. Reporting on our work in implementing such a complex set of rules encoded in Haskell's type system serves multiple purposes:

\begin{itemize}
    \item It provides a stress test for Haskell's type system, as implemented in GHC, that we hope can be used to build regression tests for the compiler's type checking performance;
    \item We report on areas of friction in developing such type-level models of complex rule sets where improvements could be made to the compiler;
    \item We identify some techniques for optimising and troubleshooting type-level code.
\end{itemize}

While our implementation can be considered idiomatic in the sense that the rules are expressed in a human-readable manner and use intuitive representations, the resulting performance does not allow us to describe longer games where memory usage becomes the limiting factor. In our testing, describing a game of around twelve moves will consume more than 20GBs of memory on the author's machine.

Reasoning about our definitions' impact on compile-time performance and memory usage has proved difficult: optimisation techniques we expected to improve performance did not and changes which we expected to make no difference ended up improving performance. These surprising findings make it clear to us that, in order for developers to be able to encode complex business logic in Haskell's type system, it must become more transparent how the compiler will react to a given definition or there must be tools which allow developers to profile and debug their encodings.

\section{Haskell Symposium 2021 Paper}

Towards the end of development, the author and their supervisor co-wrote a paper, of which Chesskell was the subject, for the ACM SIGPLAN Haskell Symposium 2021\footnote{\url{https://www.haskell.org/haskell-symposium/2021/}}. The Haskell Symposium is a well-established convention for original research on Haskell, to which papers can be submitted in an early and a regular track. These two submission tracks give authors the time to edit their papers, as feedback from the early track comes before the submission date for the regular track.

We submitted a paper to the early track, and received feedback for our submission from three anonymous reviewers; we give both the paper and the reviewer comments in \cref{symposiumfeedback}. In this section, we discuss high-level points of feedback, outlining how we will address their comments for the regular track submission.

One of the major points of feedback from all three reviewers was interest in the performance findings of Chesskell, and a desire to see more detailed analysis. While we have performed some preliminary evaluation of the performance of several games (see \cref{performanceanalysis}), there is room for greater detail. As such, before the regular track submission, we aim to perform more comprehensive performance analysis; investigating the compile time and memory usage of more games. Some moves, such as Castling and Pawn Promotion, are unlikely to be part of the opening series of moves -- and so we will broaden the testing process to include moves from the middle of games rather than just the openings of them.

This additional performance analysis will replace certain sections deemed less critical by the reviewers; such as the sections detailing the implementation of specific moves. As the paper is aimed at those already knowledgeable with type-level programming in Haskell, the techniques we explain in those sections are already known.

It should be noted that overall, the feedback was positive; while all three reviewers pointed out areas of improvement, we will address their feedback for the regular submission track and present a Chesskell paper once more.

\section{Session-typed Chesskell}

The method we have chosen to express Chess at the type-level, as a single function, is just one of many potential methods. During development, we looked into another potential method of expression; implementing Chesskell using session types~\cite{torinosessions}. Session types are formalisms of communications between two or more parties within types; that is, the types define a protocol which communicating parties must adhere to.

Interestingly, Chess could very naturally be expressed as a communications protocol between two parties, Black and White, who take turns sending messages describing their respective moves. Modelling Chess through such a protocol would enable interesting comparisons with Chesskell as it is today, and would make certain tasks easier. For instance, enforcing that teams alternate moves, and can only make one move each during their turn, would be trivial to ensure with session types.

Once Chesskell's feature set was completed, we began work on a session-typed version of Chesskell making use of an existing session types implementation~\cite{sesstypesincloudhaskell}. Early progress seemed promising; however, expressing Chesskell with this session types implementation proved difficult, due to unexpected type errors.

We created an early draft of a session-typed Chess communications protocol, using values of type \inline{Position}, before attempting to modify this trial version to use \inline{Proxy} values with type \inline{Proxy Position}. This early version involves the sender sending two positions (the destination square and the target square), and then receiving two positions from the other party, before recursively calling the function again to keep the game going for arbitrary length. We give the definition below, which compiles successfully:

\begin{lstlisting}
chess_recursion :: MonadSession m => Position -> Position
    -> m ('Cap '[] (R ((Position, Position) :!> (Position, Position)
                                            :?> Off '[V, Wk Eps])))
         ('Cap '[] Eps) ()
chess_recursion x y = recurse $ f x y
    where
        f from to = do
            send (from, to)
            pair <- recv
            (var0 >> uncurry f pair) <&> (weaken0 >> eps0)
            return ()
\end{lstlisting}

Note that the session protocol is visible in the types: \inline{R ((Position, Position) :!> (Position, Position) :?> Off '[V, Wk Eps])}. This essentially instructs the type checker to only accept session typed programs which recursively send two positions and then receive two positions.

However, the use of \inline{Cap}, \inline{R}, and \inline{V} to enable recursion through a stack of session types, which comes from a previous session types implementation in Haskell~\cite{sessalmostnoclass}, has the constraint that the types in each send-receive phase must unify with each other. This constraint makes sense; a protocol promising that an integer will be sent should fail to compile if a value of a different type is sent.

When introducing the type variables with kind \inline{'Position} into the protocol, this constraint no longer holds; there is no guarantee that a Chess game will repeat itself every few moves, and such a guarantee would rule out many valid Chess games.

Below, we use an additional type family, \inline{ModifyPos}, to generate new \inline{Position} types to mimic a Chess game. This definition fails to compile as the type variables describing the movement of the next turn of Chess will not unify with the current turn's type variables:

\begin{lstlisting}
-- The below fails to compile with the following type error:
    -- * Couldn't match type 'pos1' with 'ModifyPos pos3'
    -- 'pos1' is a rigid type variable bound by
    --   the type signature for:
    --     chess_recursion :: (...)
chess_recursion :: MonadSession m => Proxy (pos1 :: Position)
    -> Proxy (pos2 :: Position)
    -> m ('Cap ctx (R ((Proxy pos1, Proxy pos2)
        :!> (Proxy (pos3 :: Position), Proxy (pos4 :: Position))
        :?> Off '[V, Wk Eps])))
        ('Cap ctx Eps) ()
chess_recursion x y = recurse $ f x y
    where
        f from to = do
            send (from, to)
            (pair :: (Proxy p3, Proxy p4)) <- recv
            (var0 >> uncurry f
              (Proxy @(ModifyPos p3), Proxy @(ModifyPos p4)))
              <&> (weaken0 >> eps0)
            return ()

type family ModifyPos (x :: Position) :: Position where
    ModifyPos (At col row) = At col (S row)
\end{lstlisting}

The given input argument to the next round of the session typed protocol, a tuple with type \inline{(Proxy (ModifyPos pos3), Proxy (ModifyPos pos4))} cannot be used as the formal input parameter of type \inline{(Proxy pos1, Proxy pos2)} since \inline{pos1 ~ ModifyPos pos3} may not hold. However, it makes sense that the next moves are dependent on the previous moves in the game; and all moves before that point. The use of \inline{ModifyPos}, while a little crude, mimics the structure of the game, whereby each move depends upon the last.

Since this unification constraint cannot be met, expressing a Chess game with session types in this recursive manner is impossible, since there are many valid Chess games in which the moves do not repeat every $N$ terms, for some fixed positive integer $N$.

One possible solution could be to express Chess games of fixed length in this manner, simply writing new definitions for each possible Chess game length (which rarely exceeds 200). While this is technically a working solution, it is not general and would involve much shared code, complicating debugging and ensuring that if you need to remedy an issue in one definition, you must also remedy it in many other places.

Ultimately, due to time constraints, we have been unable to further investigate a potential session-typed implementation of Chesskell. However, there is room for further research into and progress within this area, perhaps through findings ways to circumvent these unification constraints through other methods of implementing recursive session typed protocols.

\chapter{Self-assessment}

This chapter details the author's self-assessment of the project; as such, it will be written in first person, and address the relevant questions directly.

\section{What is the (technical) contribution of this project?}

The primary contribution is a full type-level model of Chess in Haskell, paired with a value-level EDSL which uses the type-level model for \emph{all} rule-checking. Additionally, it provides useful points of friction for GHC, which is useful for the implementors as it points out the current limitations of type-level programming in Haskell.

% TODO: Haskell symposium paper??

\section{Why should this contribution be considered relevant and important for the subject of your degree?}

As the subject of my degree is Computer Science, any contributions towards relevant CS fields would be appropriate. The study of type systems is a broad and mature topic, and Chesskell demonstrates a complex application of academic type theory, with a view to assessing the expressivity of Haskell's type system. The achievement of implementing Chess at the type level shows knowledge of type systems, functional programming, and practical experience with Haskell and GHC; all of which are relevant to Computer Science in general, and therefore to my degree.

\section{How can others make use of the work in this project?}

Currently, Chesskell is technically usable, but is not as useful as existing Haskell Chess engines, which perform computation at the value level (and so avoid the memory issues brought about by Chesskell). Chesskell, while a practical project, is more of a proof of concept; an experiment with GHC's type level facilities. However, the type-level programming patterns used in Chesskell, chiefly First Class Families, are reusable in other projects. Furthermore, Chesskell itself could be used as a GHC stress test, checking the performance of GHC's type level computation.

\section{Why should this project be considered an achievement?}

Implementing Chess at the type level was difficult, necessitating much research, planning, and programming. Furthermore, such a thing has never been achieved before; and to my knowledge, it is one of the most complex rule sets expressed at the type level in Haskell.

\section{What are the limitations of this project?}

Currently, the severe limit to the game size (12 moves or less) is the biggest limitation, brought about by excessive memory usage. Addressing this problem would require more knowledge of GHC internals themselves, which (while not impossible) is difficult to achieve under time constraints.

Additionally, Chesskell is useful as a move checker, but currently cannot be used for many other common Chess engine operations -- such as move generation. A worthy extension could be to broaden the scope of Chesskell in this manner, making it more appropriate for use.



\section{Future Work}

Although Chesskell has a full feature set, there is of course room for further work. In this section, we detail the potential areas for improvement and research.

\subsection{Automated Chess Notation Translation}

We carried out testing of the Chesskell EDSL, as we explain in \cref{testsection}, by manually translating algebraic notation of Chess games into the Chesskell notation. A useful way to simplify this process, as well as to make Chesskell more generally usable, would be to develop an automatic tool to perform this translation.


\subsection{Type-level Bitboards} \label{bitboardconclusion}

While our implementation uses a board representation based on two-dimensional, length-indexed vectors and we investigated an equivalent representation using type-level finger trees, neither presented any significant improvements in terms of compile-time over the other. However, we believe it would be worthwhile to examine further representations, with the aim of reducing the memory footprint of GHC and the wider Chesskell type-level code.

One such potential representation is a Bitboard~\cite{bitboard}; using type-level \inline{Nat}-s to potentially reduce the amount of memory and, importantly, number of type variables that must be unified.

We could leverage type-level pattern matching to provide fast versions of "bitwise" operations on type level Nats within the range of 0 to 255 (i.e. the range that can be expressed with 8 bits). We give an example implementation of a bitwise left logical shift operation below:

\begin{lstlisting}
type family LeftShift (n :: Nat) :: Nat where
    LeftShift 0 = 0
    LeftShift 1 = 2
    LeftShift 2 = 4
    LeftShift 3 = 6
    -- ...
    LeftShift 255 = 0
    LeftShift x = TypeError (Text "Given Nat is outside of the range of 8 bits!")
\end{lstlisting}

This approach is worthy of investigation, and could bring performance improvements; we could determine if either King is under attack with a series of AND operations, rather than the current expensive implementation (see \cref{checksection}). However, it is not without flaws; it makes code considerably less readable and intuitive. The current board representation follows the natural structure of the Chess board itself, where pieces reside in one of the 64 available positions. Debugging these Nat representations would prove more difficult, since the state of the board would no longer be clear at a glance of the type.

Additionally, while it is likely to bring performance improvements, these "bitwise" operations would not map to hardware bitwise instructions, and so are still likely to be slow when compared to value-level Haskell. The additional complexity incurred through the Bitboard representation would only be worthwhile if it also brought about significant performance improvements. As such, further investigation in this area is warranted, despite its lack of inclusion in Chesskell.