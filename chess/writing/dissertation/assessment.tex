\section{Self-assessment}

This section details the author's self-assessment of the project; as such, it is written in first person singular, and addresses the relevant questions directly.

\subsection{What is the (technical) contribution of this project?}

The primary contribution is a full type-level model of Chess in Haskell, paired with value-level EDSLs which are typed with the type-level Chess model. The Haskell compiler GHC uses the type-level model to type-check games described with the EDSLs. Additionally, the project highlights points of friction for GHC performance -- which is useful for the implementors of GHC as it identifies some current limitations of type-level programming in Haskell. The final major contribution is a paper submission to the Haskell Symposium 2021, detailing notable aspects of Chesskell's development.

\subsection{Why should this contribution be considered relevant and important for the subject of your degree?}

Chesskell is useful to Computer Science as a whole due to its focus; the applied side of type systems, in Haskell and GHC specifically. As I discuss above, Haskell's type system has been the subject of much academic research and years of real-world, practical use. However, the newer, more advanced features of Haskell's type system have yet to see widespread adoption -- and so it is worthwhile to test their limits and assess how GHC copes with code that makes extensive use of them. These features will not be used in industry if they have no utility, after all; and so practical testing is a must.

\subsection{How can others make use of the work in this project?}

Chesskell includes a complete type-level model of the FIDE laws of Chess. The code is open source, so anyone who wishes to write programs which use static verification involving Chess rules (such as a typed Chess engine) can make use of Chesskell to check invariants in the types of their program. Furthermore, the benchmarks we have developed can be used by Haskell compiler implementors to assess type-checking performance. Finally, the First Class Family type-level versions of common value-level Haskell type classes (such as Monads, Foldable data structures, and Functors) are reusable and can be included in Haskell projects involving type-level programming.

\subsection{Why should this project be considered an achievement?}

Implementing Chess at the type level involved learning many techniques for type-level programming, going beyond course content and familiarising myself with state-of-the-art research in the field. Once I had learned these techniques, I had to figure out how to use them to implement Chess at the type level, which has never been achieved before; and to my knowledge, it is one of the most complex rule sets expressed at the type level in Haskell to date. Furthermore, the Haskell Symposium 2021 paper concerning Chesskell is an achievement on its own -- and the feedback on the paper is fairly positive and recognises the technical achievement of Chesskell itself.

\subsection{What are the limitations of this project?}

Currently, the limit to the game size on the author's hardware (12 moves or less) is the biggest limitation, brought about by the excessive memory usage of the GHC Haskell compiler. Addressing this problem would require improving the performance of the compiler itself, which is outside the scope of the project and would require very intimate knowledge of GHC internals. Conceptually, with more/infinitely powerful hardware, there is no limit to Chesskell game length.
