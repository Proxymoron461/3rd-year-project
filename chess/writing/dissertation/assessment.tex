\section{Self-assessment}

This section details the author's self-assessment of the project; as such, it is written in first person singular, and addresses the relevant questions directly.

\subsection{What is the (technical) contribution of this project?}

The primary contribution is a full type-level model of Chess in Haskell, paired with value-level EDSLs which are typed with the type-level Chess model. The Haskell compiler GHC uses the type-level model to type-check games described with the EDSLs. Additionally, the project highlights points of friction for GHC's performance, which is useful for the implementors of GHC as it points out some current limitations of type-level programming in Haskell.

% TODO: Write sentence on Haskell symposium paper

\subsection{Why should this contribution be considered relevant and important for the subject of your degree?}

The study of type systems is a broad and mature topic, and Chesskell demonstrates a complex application of academic type theory, with a view to assessing the expressivity of Haskell's type system. The achievement of implementing Chess at the type level shows knowledge of type systems, functional programming, and practical experience with Haskell and GHC; all of which are relevant to Computer Science in general, and therefore to my degree.

% TODO: Stress that you're looking at the applied side of type systems, in Haskell and GHC specifically
% TODO: Talk about why it's useful to Computer Science in general, not just you
% Useful because: want to push limits + assess how GHC copes with code that makes use of these features

\subsection{How can others make use of the work in this project?}

Chesskell is a complete type-level model of the FIDE laws of Chess. The code is open source, so anyone who wishes to write programs which use static verification involving Chess rules (such as a typed Chess engine) can make use of Chesskell to check invariants in the types of their program. Furthermore, the benchmarks we have developed can be used by Haskell compiler implementors to check performance of Haskell's type system.

\subsection{Why should this project be considered an achievement?}

Implementing Chess at the type level was difficult, necessitating much research, planning, and programming. Furthermore, such a thing has never been achieved before; and to my knowledge, it is one of the most complex rule sets expressed at the type level in Haskell to date.

% TODO: Explain why it was difficult: learning a lot of techniques for type-level programming (going beyond course + familiarising with state-of-the art research in the field); then, figuring out how to use those techniques to implement Chesskell (which has never been achieved before); and mention the paper! (The paper is another achievement - hard to write, got fairly positive reviews on feedback)

\subsection{What are the limitations of this project?}

Currently, the limit to the game size on the author's hardware (12 moves or less) is the biggest limitation, brought about by the excessive memory usage of the GHC Haskell compiler. Addressing this problem would require improving the performance of the compiler itself, which is outside the scope of the project and would require very intimate knowledge of GHC's internals. Conceptually, with more/infinitely powerful hardware, there is no limit to Chesskell game length.
