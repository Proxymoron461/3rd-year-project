\chapter{Introduction}

The study of programming languages in Computer Science involves, in large part, the study of type systems. Many of the interesting differences between programming languages lie not in their syntax, but in their semantics; in their behaviour. Since types govern the behaviour of languages, it is fair to say that the difference in type systems between languages forms a large part of what individuals like or dislike about programming in a specific language. Part of why assembly language can be so difficult to reason about at scale is because it is untyped; everything is a byte. C and other higher-level languages introduce \emph{type systems} for the programmer's benefit. With higher levels of abstraction, and more complex type systems, can come greater safety, as well as clearer program behaviour.

Programming languages have type systems for the main purpose of avoiding errors \cite{cardellitypes}. A \emph{type error} is an instance of attempting to perform an operation on something which does not support that specific operation. For example, it makes no logical sense to add the number 3 to a dog. This stems from the fact that "3" and "dog" support different behaviours\footnote{For instance, dogs can bark, but the number 3 cannot.}. Therefore, in a programming context, "3" and "dog" could be modelled with distinct types; 3 as a number, and a dog as an animal. By assigning types to values, programmers and the languages they use have a straightforward way to determine the valid operations on a value, and avoid type errors through misuse.

A notable area in which languages differ is \emph{when} they detect type errors. A \emph{static} type system is one in which type errors are detected before the program is run (during compilation), and a \emph{dynamic} type system is one in which type errors are detected while the program is running. A static type system is preferable for runtime safety, since it ensures that any running program will avoid (at least some) type errors.

Of course, as programming languages evolve, many have begun to address more and more errors through the type system. Features similar to optional types have been added to languages such as Java\footnote{\url{https://docs.oracle.com/javase/8/docs/api/java/util/Optional.html}} and C\#\footnote{\url{https://docs.microsoft.com/en-us/dotnet/csharp/nullable-references}}, and languages like Rust have pioneered ways of safely handling dynamic allocation through ownership types\footnote{\url{https://doc.rust-lang.org/book/ch04-01-what-is-ownership.html}}. Many compilers now force the developer to handle classes of errors that previously could only be encountered at runtime, such as null pointer exceptions.

However, \emph{logical errors}, rooted in some (typically domain-specific) behaviour, tend to nimbly evade type systems as they are not general errors that are worthy of inclusion as standard. Runtime features such as exceptions can be leveraged to discover and deal with errors and misuse of APIs. Enforcement of domain-specific invariants and rules is typically dynamic; if a check fails, an exception is thrown and potentially handled. However, if a programmer forgets to implement such a check, the program's behaviour can become unpredictable. A 2007 study \cite{exceptionsusedpoorly} on Java and .NET codebases indicates that exceptions are rarely used to recover from errors, and a 2016 analysis of Java codebases \cite{badjavaexceptions} reveals that exceptions are commonly misused in Java.

The exception misuse described above could be avoided by employing logical invariant checks at the type-level, rather than at runtime. Recent versions of the \emph{Glasgow Haskell Compiler} (GHC) support programming at the type level, allowing programmers to compute with types in the same way that languages like C or Python compute with values \cite{givingpromotion}, using \emph{type families} \cite{opentfs,closedtfs} that act as functions over types. These computations run at compile time, before the compiler generates an executable of the source code, allowing programmers to transform logical errors into type errors \cite{twt}.

A growing number of new languages have type systems which support \emph{Dependent types}, in which the types themselves depend on runtime values, and can be treated as values. The programming language Idris is similar to Haskell, but allows the programmer to pass around types at runtime, and write functions which operate on those types. Many of Haskell's language extensions have been adding to its type system, moving the language closer and closer towards dependently typed programming \cite{singletons}. Such a type system has obvious benefits, since constraining the types means constraining the values without dynamic runtime checks. (For example, in a dependently typed environment, runtime array bounds checks can be eliminated at runtime through being expressed solely in the type system \cite{dependentarray}.) However, Haskell is not at that stage yet; it lacks a fully dependent type system. Instead, these extensions must be used in creative ways to mimic such behaviour.

Since these are relatively recent developments, there are few examples of their usage in complex applications. For developers to adopt such techniques and express sophisticated invariants in the types of their program, these GHC extensions should not negatively impact development experience; particularly compile time. In this project, to stress-test GHC with a complex set of invariants, we show how to utilise type-level programming features in Haskell in order to model the classic board game Chess in Haskell's type system, ruling out invalid moves with types. A Haskell-Embedded Domain-Specific Language (DSL), for describing games of Chess, will interact with the type-level model. This Embedded DSL (EDSL) will be modelled on Algebraic Notation, a method of writing down the moves associated with a particular match of Chess. We implement the full, official International Chess Federation (FIDE) ruleset for Chess.

\section{Related Work}

Chesskell is, at the time of writing, unique; we are aware of no other type-level Chess implementations. There have been allusions to Chess at the type-level through solving the N-queens problem in dependently typed languages, such as Idris\footnote{\url{https://github.com/ExNexu/nqueens-idris}}. The N-queens problem makes use of some Chess rules, including the Queen's attack positions\footnote{A Queen can attack in a straight line in any direction.}; but as the end goal is not to successfully model a game of Chess, it is not a full type-level Chess implementation.

However, Chesskell draws from, and owes much to, many well-established research areas, including type-level rule checking, EDSLs, and Chess programming in general. This section of the report will detail related work, and how Chesskell differs from existing literature.

\subsection{Type-level Rule Checking}

The idea of using types to enforce rules on behaviour is hardly specific to Haskell; after all, C and C-like languages attempt to ensure that you only apply specific operations on values of certain types. The programming language Rust\footnote{\url{https://www.rust-lang.org/}} has been voted the most loved language (by StackOverflow developers) 5 years running\footnote{\url{https://insights.stackoverflow.com/survey/2020\#technology-most-loved-dreaded-and-wanted-languages}}. Rust is touted as a systems language that guarantees memory safety and thread safety; and it achieves this through its type system. By enforcing strict ownership rules, Rust can guarantee that compiling programs avoid data races and that all memory is freed once (and not used after being freed). This is a clear example of types enforcing runtime behaviour; a series of memory safety rules are being enforced. In fact, Haskell type-level constructs can be used to enforce basic ownership rules through a method colloquially known as the "ST Trick" \cite{twt}.

Of course, type-level rule checking in Haskell is very possible. Through clever use of types, Lindley and McBride's merge sort implementation \cite{hasochism} is guaranteed to produce sorted outputs. Unit tests for the sorting implementation become unnecessary, since the GHC type checker is used to ensure that the sort itself behaves correctly. The type system is leveraged to enforce the rule that sorted data should be in sort order.

The above demonstrate the fact that type-level behaviour enforcement is neither new nor specific to Haskell. Though type systems can be complex, since many languages are designed to be general-purpose their type systems are also designed to be so. Chesskell represents an attempt at capturing domain-specific knowledge at the type level, and using that knowledge to maintain safe behaviour. Chesskell, and other type-level behaviour enforcers, are not common simply because logic errors are usually dealt with through dynamic checks (it is certainly easier to write dynamic unit tests than it is to model your application domain with types).

\subsection{Haskell-Embedded Domain-Specific Languages}

Despite the apparent lack of work on Chess at the type level specifically, there is work on Haskell-Embedded DSLs in other domains to enforce certain behaviour at compile time. DSLs exist for the purpose of modelling some domain in a language; so Haskell-Embedded DSLs are a natural use case for domain-specific modelling with types. If an EDSL comes with the guarantee that all compiling programs written in that language will not exhibit invalid behaviour, then the EDSL becomes an attractive way to interact with that domain.

Mezzo \cite{mezzohaskellsymposium} is an EDSL for music composition, which checks if the described piece of music conforms to a given musical ruleset during compilation of the program. For instance, one can apply classical harmony rules to ensure that the piece of music you compose would not go against the rules of the musical period. This EDSL is similar to Chesskell in aim, if not in application domain; performing compile-time checks of rulesets that are commonly checked dynamically. Mezzo is an example of a complex domain with complex rules (classical harmony) being modelled and enforced at the type-level. This is similar to Chesskell's objectives, and was a direct inspiration for the project.

As another example, BioShake \cite{bioshake} is an EDSL for creating performant bioinformatics computational workflows. The correctness of these workflows is checked during compilation, preventing any from being created if their execution would result in certain errors. For bioinformatics workflows especially, this is ideal since many of these workflows are lengthy. BioShake goes further, however; providing tools to allow parallel execution of these workflows. While it is encouraging to see BioShake and other EDSLs \cite{aplite} focus on (and achieve) high performance, Chesskell has no such focus. This is primarily because very few parts of the rule-checking process can be parallelised; much of the move handling and order of rule checks must be done sequentially.

\subsection{Chess in Computer Science}

Chess has a rich history as a study area of Computer Science. Getting computers to play Chess was tackled as far back as 1949 \cite{1949chess}, and since then many developments have been made in the field. Chess has been used to educate \cite{chesseducation}, to entertain, and to test out machine learning approaches \cite{chessml}. Due to its status as a widely known game of logic, with a well-defined rule set, it is a prime candidate to act as a general setting for programming problems. Indeed, the famous NP-Complete problem referenced above, the N-Queens Problem \cite{nqueensnp}, relies on the rules of Chess.

Many of these Chess-related programs are written in Haskell, and are publicly available\footnote{\url{https://github.com/mlang/chessIO}}\footnote{\url{https://github.com/nionita/Barbarossa}}. A large number are Chess engines, which take in a board state and output the move(s) which are strongest, and so therefore perform move checking at the value-level to ensure that the generated moves are valid. Chesskell differs from these in function, in that the end software does not output a list of strong moves; it simply takes in the moves performed, and state whether they are valid Chess moves or not. We are not aware of any such type-level Chess implementations in Haskell, or any other language.

Game development, as a more general field in Computer Science, has many Chess-based or Chess-related games available. However, the intention in these cases is usually to facilitate real-time play between multiple players (or indeed a single player with a competitive AI), rather than to teach or program a machine to consistently beat players. There is overlap with Chesskell; Chess as a computer game must necessarily perform move validation (to disallow cheating) and ensure that players take turns. However, Chesskell is intended to check over a complete game, rather than to enable people to conduct a game in real-time with Chesskell as a mediator.

\subsection{Why Haskell?}

Given the previous discussion on dependent type systems, and how Haskell is inching towards one, it begs the question; why not use a dependently typed language, like Idris, to write Chesskell in? The simple answer is because it would be trivial. Writing type-level code in Idris (or any other dependently typed language) would be near indistinguishable from writing a Chess validity checker, bundled within an EDSL, at the value-level. Such a feat is both simple and unoriginal.

However, choosing to write Chesskell in Haskell means figuring out how to perform typically value-level computation at the type-level. Indeed, the majority of the code for Chesskell is reusable, since much of it is not specifically about expressing the rules of Chess, but building components to enable complex computation with types. The project is both more difficult, and more interesting, for having been completed in a language without a fully dependent type system.

\section{Objectives}

The objective of this project is to develop a model of Chess at the type level, which will compile a given program if and only if it is a valid game of Chess. The primary method of interfacing with this type level model will be via a custom EDSL, through which Chess games are expressed. During compilation, the game of Chess will be simulated, such that any invalid move (or the lack of a move where one should have occurred) will result in a type error. The main goals are thus:

\begin{itemize}
    \item Develop a type-level model of a Chess board;
    \item Develop a type-level move-wise model of a Chess game;
    \item Develop an EDSL to express these type-level Chess games in, which uses the type-level model of Chess to validate moves;
    \item Ensure (through testing) that valid Chess games compile, and invalid Chess games do not.
\end{itemize}

We define a "valid Chess game" as any game which adheres to the FIDE 2018 Laws of Chess\footnote{\url{https://handbook.fide.com/chapter/E012018}}. The FIDE laws also contain rules for the players themselves to adhere to; but these are outside the scope of the project, since they are not directly concerning the game of Chess itself.
