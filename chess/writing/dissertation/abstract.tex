\phantomsection
\addcontentsline{toc}{chapter}{Abstract}
\begin{abstract}
    Type-level programming, a relatively recent phenomenon, allows programmers to express computation during the compilation of their programs. Through the use of type-level constructs, rules can be imposed on code to ensure that if it compiles, then it behaves in a certain way. However, there is still plenty of room to push the boundaries of what can be achieved with type-level programming.

    Chess has a well-defined ruleset, and has not been expressed at the type level before. This dissertation describes the development of Chesskell, a full type-level model of, and rule-checker for, Chess---along with a Haskell-Embedded Domain-Specific Language for notating Chess games. If the Chesskell code compiles, then the match described obeys the full International Chess Federation ruleset for Chess. Despite difficulties during development, including memory issues, the final version of Chesskell is feature-complete and supports Chess games of up to 12 moves.

    \textbf{\textit{Keywords:}} Type-level Programming, Haskell, Chess, EDSL.
\end{abstract}